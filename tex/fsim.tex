\documentclass[rocket.tex]{subfiles}

\begin{document}

\subsection{Estructura}

\subsubsection{Integraci�n Num�rica}

\paragraph{1DOF}

\begin{equation*}
	\begin{bmatrix} \dot{x}_1 \\ \dot{x}_2 \end{bmatrix}
	=
	\begin{bmatrix} 0 & 1 \\ -a_1 & -a_2 \end{bmatrix}
	\begin{bmatrix} x_1 \\ x_2 \end{bmatrix}
	+
	\begin{bmatrix} 0 \\ b \end{bmatrix}
	\begin{bmatrix} u \end{bmatrix}
\end{equation*}

\begin{align*}
	\Phi(s) = \left[s \mtx{I} - \mtx{A}\right]^{-1} 
	&= 
	\frac{1}{s^2 + a_2 s + a_1}
	\begin{bmatrix} a_2 + s & 1 \\ -a_1 & s  
	\end{bmatrix}
	\\
	\Phi(s) \mtx{B} 
	&= 
	\frac{1}{s^2 + a_2 s + a_1}
	\begin{bmatrix} b \\ bs  
	\end{bmatrix}
\end{align*}

\begin{align*}
	\mathcal{L}\left\{
	\frac{1}{s^2 + a_2 s + a_1}
	\right\}
	&= \frac{1}{\omega_n \sqrt{1 - \xi^2}} \cdot e^{-\xi \omega_n t} \cdot \sin \left(\sqrt{1 - \xi^2} \omega_n t \right)
	 \\ 
	 a_1 &= \omega_n^2 \quad , \quad a_2 = 2\xi \omega_n
\end{align*}


\subsubsection{Ejecuci�n}


\subsubsection{Post-procesamiento}





\subsection{Modelo}

\subsubsection{Aerodin�mica}


\subsubsection{Populsi�n}


\subsubsection{Actuadores}


\subsubsection{Sensores}






\subsection{Parametrizaci�n}

\paragraph{Par�metros M�sicos: \texttt{S\#\_mass.csv}}

Tabla ASCII con filas:

\begin{equation*}
	m\ ,\ x_{cg}\ ,\ y_{cg}\ ,\ z_{cg}\ ,\ I_x\ ,\ I_y\ ,\ I_z
\end{equation*}

\paragraph{Masa Expulsable: \texttt{S\#\_mexp.csv}}

Tabla ASCII con una fila:

\begin{equation*}
	m\ ,\ x_{cg}\ ,\ I_x\ ,\ I_y\ ,\ I_z
\end{equation*}

\paragraph{Par�metros El�sticos: \texttt{S\#\_flex.csv}}

Tabla ASCII con columnas:

\begin{equation*}
	m\ ,\ \omega_n\ ,\ \xi\ ,\ \eta_w\ ,\ \eta_q\ ,\ \eta_f\ ,\ y(x_{isa})\ ,\ y'(x_{isa})
\end{equation*}

\paragraph{Par�metros de Propulsi�n: \texttt{S\#\_prop.csv}}

Archivo ASCII con una columna:

\begin{tabular}{l | l}
	$\dot{m}$ & caudal m�sico \\
	$p_e$   & presi�n de escape \\
	$A_e$   & �rea de escape \\
	$x_{gimbal}$ & posici�n del guimbal (respecto de la proa) \\
	$\delta_{max}$ & deflexi�n m�xima del TVC \\
	$p_a[1]$ & valores de presi�n atmosf�rica \\
	$p_a[2]$ & \\
	$\ \ \vdots$ & \\
	$p_a[n]$ & \\
	$v_e[1]$ & valores de velocidad de escape \\
	$v_e[2]$ & \\
	$\ \ \vdots$ & \\
	$v_e[n]$ &  
\end{tabular}

\paragraph{Par�metros Aerodin�micos: \texttt{S\#\_aero.dat}}

Archivo binario con datos de acuerdo al Ap�ndice \ref{apnd:aero}.

\subsubsection{Configuraci�n de la FCC}

\paragraph{Plan de Vuelo: \texttt{params.ini}}

Par�metros para configurar el plan de vuelo: 

\begin{lstlisting}[xleftmargin=3cm] 

[times]
sample     = 0.01
S1 burn    = 121.1
S2 burn    = 299.4
pre-meco   = 2
S1 coast   = 20
separation = 1
S2 fire    = 2

[heights]
relief = 4
vaccum = 84
steer  = 70
separation = 70

\end{lstlisting}

\paragraph{Control Autom�tico: \texttt{params.ini}}

Ganancias para los lazos de control (\texttt{gains\_S1/gains\_S2}):
\bigskip

\begin{lstlisting}[xleftmargin=3cm]

[gains_S1]
pitch_f  = 2
pitch_g  = 2
pitch_L  = 0
pitch_k2 = 0
pitch_rate_limit = 0.5

roll_f  = 0.01
roll_g  = 1
roll_L  = 0
roll_k2 = 0
roll_rate_limit = 0.5

\end{lstlisting}

%[gains_S2]
%pitch_f  = 2
%pitch_g  = 1
%pitch_L  = 0.05
%pitch_k2 = 0
%pitch_rate_limit = 0.1
%
%roll_f  = 1
%roll_g  = 90
%roll_L  = 0
%roll_k2 = 0
%roll_rate_limit = 0.5
%\end{lstlisting}

\paragraph{Guiado Activo: \texttt{params.ini}}

Par�metros para el guiado activo (LTG y Terminal):

\begin{lstlisting}[xleftmargin=3cm]

[guidance]
pre_cycle = 1
tgo_min   = 4
v_exit    = 3202
v_epsilon = 1
min_sfc   = 0.4

[orbit]
height      = 180
inclination = 39
ascending node = 0

\end{lstlisting}

\subsubsection{Simulaci�n: \texttt{params.ini}}

Par�metros espec�ficos de la simulaci�n:

\begin{lstlisting}[xleftmargin=3cm]

[space port]
latitud  = -38.467314
longitud = -58.308232 

[simulation]
integration time = 0.001
duration         = 500
start time       = 10
launch time      = 30
model root       = data/S
export root      = rec/orb_40wx1cm_2st
wind profile     = data/wind_4.csv
separation rot   = 4 10 0
max angular rate = 180

\end{lstlisting}




\end{document}
