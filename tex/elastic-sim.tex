\documentclass[rocket.tex]{subfiles}

\begin{document}

\subsection{Modos el�sticos}

La din�mica de un modo el�stico lineal puede expresarse mediante un modelo de estados de la forma:
\begin{equation}
	\begin{Bmatrix} \dot{\eta}_j \\ \dot{v}_j \end{Bmatrix}
	=
	\begin{bmatrix} 0 & 1 \\ -\omega_j^2 & -2 \xi_j \omega_j  \end{bmatrix}
	\begin{Bmatrix} \eta_j \\ v_j \end{Bmatrix}
	+
	\begin{bmatrix} 0 \\ b_j \end{bmatrix}
	f_j 
\end{equation}
donde $v_j = \dot{\eta}_j$ y $b_j = m_j^{-1}$.
Para los modelos lineales puede obtenerse una soluci�n exacta, de lo cual, para tiempo discreto se obtiene:
\begin{equation}
	\vect{x}_{k+1} = \mtx{\Phi}(t_s)\, \vect{x}_k + \int_0^t \mtx{\Phi}(t_s) dt\, \mtx{B}\, \vect{u}_k 
\end{equation}
donde $\mtx{\Phi}(t)$ es la matriz de transici�n de estados.
Para encontrarla aplicamos transformada de La place al modelo de tiempo continuo:
\begin{equation}
	\begin{Bmatrix} \eta_j(s) \\ v_j(s) \end{Bmatrix}
	=
	\mtx{\Phi}(s) 
	\left[
	\begin{Bmatrix} \eta_j(0) \\ v_j(0) \end{Bmatrix}
	+
	\begin{bmatrix} 0 \\ b_j \end{bmatrix}
	f_j(s) 
	\right]
\end{equation}
donde $\mtx{\Phi}(s)$es la Transformada de Laplace de la Matriz de Transici�n de Estados
\begin{equation}
	\mtx{\Phi}(s) 
	=
	\frac{1}{s^2 + 2\xi\omega s + \omega^2}
	\begin{bmatrix} s+2\xi\omega & 1 \\ -\omega^2 & s \end{bmatrix}
\end{equation}
Anti-transformando:
\begin{equation}
	\mtx{\Phi}(t) =
	\begin{bmatrix} \phi_{11}(t) & \phi_{12}(t) \\ \phi_{21}(t) & \phi_{22}(t) \end{bmatrix}
\end{equation}
donde:
\begin{align}
	\phi_{21}(t) &= -\omega / \hat{\xi}\, e^{-r t} \sin\left(\omega_d t\right) \\
	\phi_{22}(t) &= -e^{-r t}/\hat{\xi} \, \sin\left(\omega_d t - \beta\right) \\
	\phi_{12}(t) &= -\phi_{21}(t)/\omega^{2}  \\
	\phi_{11}(t) &= \phi_{22}(t) + r\, \phi_{12}(t) 
\end{align}
con las sustituciones:
\begin{align}
	r = \xi\omega     \quad , \quad 
	\omega_d = \omega \sqrt{1 - \xi^2}    \quad , \quad 
	\hat{\xi} = \sqrt{1 - \xi^2}     \quad , \quad 
	\beta = \tan^{-1}\hat{\xi}/\xi
\end{align}
Para la matriz de entradas se debe calcular su integral:
\begin{align}
	\int \phi_{22}(t) dt &= \phi_{12}(t) \\
	\int \phi_{21}(t) dt &= e^{-r t}/\hat{\xi} \, \sin\left(\omega_d t + \beta\right) - 1\\
	\int \phi_{12}(t) dt &= -\int \phi_{21}(t) dt /\omega^{2}  \\
	\int \phi_{11}(t) dt &= \int \phi_{22}(t) dt + r\, \int \phi_{12}(t) dt 
\end{align}


La influencia de la deformaci�n el�stica y sus derivadas sobre distintas variables de salida debe determinarse mediante la forma modal correspondiente, teniendo en cuenta que:



\end{document}
