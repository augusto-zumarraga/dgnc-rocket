\documentclass[rocket.tex]{subfiles}

\begin{document}

\subsection{Navegaci�n}

Las ecuaciones de navegaci�n son no-lineales, por lo cual se propone una integraci�n num�rica por Runke-Kutta de orden 4.


%Utilizamos la ecuaci�n \eqref{ecu:tray_v_dot} para los cambios en la velocidad, pero para la senda de ascenso $\gamma$ debemos agregar la la fuerza normal a la \eqref{ecu:tray_gamma_dot} cuando $\alpha \neq 0$.
%Para el �ngulo de ataque se tiene que:
%\begin{equation*} 
%	\alpha = \sin^{-1} \frac{w^b}{V} \approx \frac{w^b}{V}
%\end{equation*}
%y para el �ngulo de la velocidad:
%\begin{equation*} 
%	\gamma = \theta - \alpha 
%\end{equation*}
%con lo cual
%\begin{equation*} 
%	\dot{\gamma} = \dot{\theta} - \dot{\alpha} 
%	%           \\&= q - \frac{\partial \alpha}{\partial w} \dot{w} - \frac{\partial \alpha}{\partial V}\dot{V}
%	           = q - \frac{1}{V} \dot{w} + \frac{w}{V^2}\dot{V}
%\end{equation*}
%Y como $w = V \sin \theta$:
%\begin{equation*} 
%	\dot{\gamma} = q - \frac{1}{V} \left(\dot{w} - \dot{V} \sin \theta \right)
%\end{equation*}
%De la suma de fuerzas en $z^b$ tenemos que:
%\begin{equation*} 
%	\dot{w} = g \cos \theta - f_N \sin \alpha
%\end{equation*}
%con lo cual finalmente:
%\begin{equation*} 
%	\dot{\gamma} = q - \frac{1}{V} \left(g \cos \theta - f_N \sin \alpha - \dot{V} \sin \theta \right)
%\end{equation*}


%donde $w$ es la componente vertical de la velocidad (con eje $z^e$ apuntando hacia abajo)

\end{document}
