\documentclass[rocket.tex]{subfiles}

\begin{document}

\subsection{Modelo en Coordenadas Modales}

Cualquier modelo mec�nico lineal de N grados de libertad puede escribirse de la forma:
\begin{equation*}
	\mtx{M}\ddot{\vect{q}} + \mtx{K} \vect{q} = \mtx{F} \vect{u} 
\end{equation*}
o bien 
\begin{equation*}
	\ddot{\vect{q}} + \bar{\mtx{K}} \vect{q} = \bar{\mtx{F}} \vect{u} 
	\qquad , \qquad 
	\bar{\mtx{K}} = \mtx{M}^{-1}\mtx{K} 
	\qquad , \qquad 
	\bar{\mtx{F}} = \mtx{M}^{-1}\mtx{F} 
\end{equation*}
donde $\vect{q}$ es el vector de coordenadas generalizadas y $\vect{u}$ el de perturbaciones. 
Esto puede llevarse a coordenadas modales desacopladas utilizando como transformaci�n la matriz modal $\mtx{\Lambda}$ de $\bar{\mtx{K}}$, es decir, sustituir:
\begin{equation}\label{ecu:generalized-to-modal}
	\vect{q} = \mtx{\Lambda} \vect{\eta}
\end{equation}
lo que resultar� de la forma:
\begin{equation}\label{ecu:perturbed-modal-mtx}
	\left\{\mtx{\Lambda}\ddot{\vect{\eta}}\right\} + \bar{\mtx{K}} \left\{\mtx{\Lambda} \vect{\eta}\right\} = \bar{\mtx{F}} \vect{u}
	\qquad \rightarrow \qquad
	\ddot{\vect{\eta}} + \mtx{\lambda} \vect{\eta} = \mtx{P} \vect{u}
\end{equation}
donde $\mtx{P} = \mtx{\Lambda}^{-1} \bar{\mtx{F}}$ y la matriz $\mtx{\lambda}$ es diagonal. 

De forma expandida en coordenadas modales la ecuaci�n matricial de segundo orden resulta:
\begin{align*}%\label{ecu:modal-coordinates-model}
	\begin{Bmatrix} \ddot{\eta}_1 \\ \ddot{\eta}_2 \\ \vdots \\ \ddot{\eta}_n \end{Bmatrix} 
	+
	\begin{bmatrix}
		\lambda_1 & 0 & \cdots & 0 \\
		0 & \lambda_2 & \cdots & 0 \\
		\vdots & \vdots & \ddots & \vdots \\
		0 & 0 & \cdots & \lambda_n
	\end{bmatrix}  
	\begin{Bmatrix} \eta_1 \\ \eta_2 \\ \vdots \\ \eta_n \end{Bmatrix}  
	=
	\begin{bmatrix}
		p_{11} & \cdots & p_{1m}\\
		p_{21} & \cdots & p_{2m}\\
		\vdots       & \ddots & \vdots \\
		p_{n1} & \cdots & p_{nm}
	\end{bmatrix}
	\begin{Bmatrix} u_1 \\ \vdots \\ u_m \end{Bmatrix}
\end{align*}
El lado derecho expresa el efecto de las las fuerzas generalizadas sobre las aceleraciones modales. 
Una fila de la matriz $\mtx{P}$ indica como las perturbaciones se combinan para estimular un determinado modo natural. 
\bigskip

Con el modelo en coordenadas modales se puede obtener un modelo de estados a partir de la ecuaci�n \eqref{ecu:state-ndof}, 
de la misma forma que se har�a con la ecuaci�n matricial de segundo orden original.

Para las salidas la forma general de la ecuaci�n correspondiente para un modelo din�mico lineal es:
\begin{equation}
	\vect{y} = \mtx{C}\vect{x} + \mtx{D}\vect{u} 
\end{equation}
Si solo nos interesa observar deformaciones o solicitaciones estructurales:
\begin{equation}
	\vect{y} = \mtx{C}\vect{q} 
\end{equation}
Al llevar el modelo din�mico a coordenadas generalizadas se usa la sustituci�n \eqref{ecu:generalized-to-modal} tambi�n para la ecuaci�n de salidas:
\begin{equation}\label{ecu:modal-def-output}
	\vect{y} = \bar{\mtx{C}}\vect{\eta} 
	\qquad , \qquad 
	\bar{\mtx{C}} = \mtx{C}\mtx{\Lambda} 
\end{equation}

\subsection{Modelo de Orden Reducido}

Agrupando las coordenadas modales en dos vectores $a$ y $b$ se puede organizar la ecuaci�n \eqref{ecu:perturbed-modal-mtx} en la forma:
\begin{align*} 
	\begin{Bmatrix} \ddot{\vect{\eta}}_a \\ \ddot{\vect{\eta}}_b \end{Bmatrix} 
	+
	\begin{bmatrix} \vect{\lambda}_a & 0 \\ 0 & \vect{\lambda}_b \end{bmatrix}  
	\begin{Bmatrix} \vect{\eta}_a \\ \vect{\eta}_b \end{Bmatrix}  
	=
	\begin{bmatrix} \mtx{P}_a \\ \mtx{P}_b \end{bmatrix}
	\vect{u} \\
	\vect{y} = \begin{bmatrix} \mtx{C}_a & \mtx{C}_b \end{bmatrix}
	           \begin{Bmatrix} \vect{\eta}_a \\ \vect{\eta}_b \end{Bmatrix}  
\end{align*}
siendo $\vect{\lambda}_a$ y $\vect{\lambda}_b$ submatrices diagonales.
Si en $b$ se incluyen los modos que puedan considerarse de alta frecuencia para un determinado an�lisis, es posible despreciar su din�mica. 
Esto equivale a asumir $\ddot{\vect{\eta}}_b \approx 0$ y reducir el modelo a:
\begin{align*} 
	\ddot{\vect{\eta}}_a + \vect{\lambda}_a \vect{\eta}_a
	=
	\mtx{P}_a \vect{u}
	\qquad , \qquad 
	\vect{\eta}_b = \vect{\lambda}_b^{-1} \mtx{P}_b \vect{u}
\end{align*}
El t�rmino de la derecha contempla la contribuci�n a la deformaci�n est�tica de los modos de alta frecuencia.
Debe notarse las filas de la matriz $\mtx{P}_b$ para aquellas variables de salida que representen velocidades o aceleraciones ser�n nulas, 
ya que este t�rmino da cuenta solo del aporte de los modos de alta frecuencia en r�gimen estacionario ($\vect{\dot{q}} = \vect{\ddot{q}} = 0$).

La ecuaci�n de salida para desplazamientos resulta entonces:
\begin{align*} 
	\vect{y} = \mtx{C}_a \vect{\eta}_a + \mtx{C}_b \vect{\lambda}_b^{-1} \mtx{P}_b \vect{u}
\end{align*}
El modelo de estados reducido queda:
\begin{align}\label{ecu:reduced-mec-mdl} 
	\begin{Bmatrix} \dot{\vect{\eta}}_a \\ \ddot{\vect{\eta}}_a \end{Bmatrix} 
	&=
	\begin{bmatrix} \mtx{0} & \mtx{I} \\ -\vect{\lambda}_a & \mtx{0} \end{bmatrix}  
	\begin{Bmatrix} \vect{\eta}_a \\ \vect{\dot{\eta}}_a \end{Bmatrix}  
	+
	\begin{bmatrix} \mtx{0} \\ \mtx{P}_a \end{bmatrix}
	\vect{u} 
	\\ \nonumber 
	\vect{y} &= \begin{bmatrix} \mtx{C}_a & \mtx{0} \end{bmatrix}
			    \begin{Bmatrix} \vect{\eta}_a \\ \vect{\dot{\eta}}_a \end{Bmatrix}  
			  + \left[\mtx{C}_b \vect{\lambda}_b^{-1} \mtx{P}_b\right] \vect{u}
\end{align}
Si el inter�s se centra cierta en alguna combinaci�n de velocidades se tendr�a que:
\begin{align*} 
	\vect{y} = \mtx{C}_a \vect{\dot{\eta}}_a =
	\vect{y} = \begin{bmatrix} \mtx{0} & \mtx{C}_a \end{bmatrix}
	\begin{Bmatrix} \vect{\eta}_a \\ \vect{\dot{\eta}}_a \end{Bmatrix}  
	+ \mtx{0}\ \vect{u}
\end{align*}
y si se tratase de aceleraciones:
\begin{align*} 
	\vect{y} = \mtx{C}_a \vect{\ddot{\eta}}_a 
	         = - \begin{bmatrix} -\mtx{C}_a \vect{\lambda}_a & \mtx{0} \end{bmatrix}
	             \begin{Bmatrix} \vect{\eta}_a \\ \vect{\dot{\eta}}_a \end{Bmatrix}  
	             + \left[\mtx{C}_a \mtx{P}_a\right] \vect{u}  
\end{align*}

%\pagebreak
%
%\begin{shaded}
%
%Supongamos que ya contamos con las matrices de masa $\mtx{M}$ y rigidez $\mtx{K}$, la matriz de perturbaci�n $\mtx{F}$ y la de salida $\mtx{C}$.
%Podemos calcular la matriz modal y realizar el cambio de coordenadas:
%
%\begin{lstlisting}[language=Matlab, xleftmargin=2.5cm, basicstyle=\small]
%K_ = M\K;
%[V,L] = eig(K_, 'vector');
%
%P = V\(M\F); % cuidado con el orden deL operador '\'
%C = C*V;
%\end{lstlisting}
%
%Podemos ordenar los modos naturales en orden creciente de frecuencias, 
%y particionar las matrices para construir un modelo reducido de orden $nr$:
%
%\begin{lstlisting}[language=Matlab]
%% ordenamos de menos a mayor, y en p guardamos los �ndices (filas) 
%% del correspondiente ordenamiento original 
%[~,p] = sort(L); 
%
%ka = p(1:nr);     % indices para los modos de baja frecuencia 
%kb = p(nr+1:end); % indices para los modos de alta frecuencia 
%
%La = diag(L(ka)); % matrices lambda de baja y alta frecuencia
%Lb = diag(L(kb));
%Pa = P(ka,:);     % matrices de perturbaci�n para modos 
%Pb = P(kb,:);     % de baja y alta frecuencia
%Ca = C(:,ka);     % relaci�n entre las salidas y los modos 
%Cb = C(:,kb);     % de baja y alta frecuencia
%\end{lstlisting}
%
%Con esto podemos construir el modelo de estados de orden reducido:
%\begin{lstlisting}[language=Matlab]
%A  = [zeros(nr) eye(nr)
%     -La        zeros(nr)];
%B  = [zeros(size(Pa))
%      Pa];
%C  = [Ca zeros(size(Ca))];
%D  = Cb*(Lb\Pb);
%Me = ss(A, B, C, D);
%\end{lstlisting}
%
%\end{shaded}

\end{document}
